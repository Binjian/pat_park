%!TEX program = <xelatex>
%!TEX output_directory = <K:\LaTeX>
%!TEX aux_directory = <K:\LaTeX\aux>
%!TEX jobname = <test_xelatex> 
\documentclass[12pt,a4paper]{article}
\usepackage{xeCJK}
\usepackage{latexsym}
\usepackage{amsmath}                 % AMS LaTeX宏包
\usepackage{amssymb}                 % 用来排版漂亮的数学公式
\usepackage{amsbsy}
\usepackage{amsthm}
\usepackage{amsfonts}
\usepackage{mathrsfs}                % 英文花体字体
\usepackage{bm}                      % 数学公式中的黑斜体
\usepackage{relsize}                 % 调整公式字体大小:\mathsmaller, \mathlarger
\usepackage{cmap}                   % 使pdfLatex生成的文件支持复制等
\usepackage{graphicx}                % 用于图像
\usepackage{caption}
\usepackage{setspace}                % 调节行间距
\usepackage{booktabs}                % 用于表格中加下划线
\usepackage{fancyhdr}                % 页眉页脚
\usepackage{type1cm}                 % 控制字体大小
\usepackage{indentfirst}             % 首行缩进
\usepackage{makeidx}                 % 建立索引
\usepackage{textcomp}                % 千分号等特殊符号
\usepackage{layouts}                 % 打印当前页面格式
\usepackage{bbding}                  % 一些特殊符号
\usepackage{cite}                    % 支持引用
\setlength{\skip\footins}{0.5cm}     % 脚注与正文的距离
%%%%%%%%%%%%%%%%%%%%%%%%%%以上是版面控制部分%%%%%%%%%%%%%%%%%%%%%%%%%%%%%%%%%%%%%%%%%%%%%

%%%%%%%%%%%%%%%%%%%%%%%以下是版面控制部分%%%%%%%%%%%%%%%%%%%%%%%%%%%%%%%%%%%%%%%%%%%%%%
\usepackage{geometry}\geometry{left=2.75cm,right=2.5cm,top=2.5cm,bottom=2.5cm}
\usepackage{indentfirst}             % 首行缩进
\usepackage[perpage,symbol]{footmisc}% 脚注控制
\usepackage[sf]{titlesec}            % 控制标题
\usepackage{titletoc}                % 控制目录
\titlecontents{section}[0pt]{\addvspace{2pt}\filright}
              {\contentspush{\thecontentslabel\ }}
              {}{\titlerule*[8pt]{.}\contentspage}
                                     % 添加section在目录里的点号




%%%%%%%%%%%%%%%%%%%%%%%%%以下为中英文字体设置%%%%%%%%%%%%%%%%%%%%%%%%%%%%%%%%%%%%%%%%%%%
\usepackage{times}
\usepackage{fontspec,xunicode,xltxtra} % XeLaTeX相关字体字库
\XeTeXlinebreaklocale "zh"
\XeTeXlinebreakskip = 0pt plus 1pt minus 0.1pt
%\newfontfamily\youyuan{YouYuan}
%\newfontfamily\hwcaiyun{STCaiyun}
%\newfontfamily\hwhupo{STHupo}
%\newfontfamily\yaoti{FZYaoTi}
%\newfontfamily\kaiti{KaiTi_GB2312}

\newfontfamily\xsong{SimSun}
%\newfontfamily\hwsong{STSong}
\newfontfamily\yahei{SimHei}
%\newfontfamily\fangsong{FangSong_GB2312}
\newfontfamily\song{AdobeSongStd-Light}
%\newfontfamily\hwfangsong{STFangsong}
%\newfontfamily\weiti{STXinwei}
\newfontfamily\hei{AdobeHeitiStd-Regular}
%\newfontfamily\hwxingkai{STXingkai}
%\newfontfamily\hwlishu{STLiti}
%\newfontfamily\zhongsong{STZhongsong}
%\newfontfamily\shuti{FZShuTi}
%\newfontfamily\hwhei{STXihei}
%\newfontfamily\lishu{LiSu}
%\newfontfamily\hwkai{STKaiti}
\newfontfamily\tnroman{Times New Roman}
\newfontfamily\consol{Consolas}
\newfontfamily\li{LiSu}
\newcommand{\sanhao}{\fontsize{16pt}{24pt}\selectfont}      % 三号, 1.5倍行距
\newcommand{\sihao}{\fontsize{14pt}{21pt}\selectfont}       % 四号, 1.5倍行距
\newcommand{\xiaosi}{\fontsize{12pt}{18pt}\selectfont}      % 小四, 1.5倍行距
\newcommand{\wuhao}{\fontsize{10.5pt}{10.5pt}\selectfont}   % 五号, 单倍行距
\setCJKmainfont{SimSun}   % 设置默认中文字体



%%%%%%%%%%%%%%%%%%%%%%%%%%%%%%以下是一些命令或环境的重定义或自定义%%%%%%%%%%%%%%%%%%%%%%
\newtheorem{theorem}{定理}
\newtheorem{definition}{定义}
\newtheorem{property}{问题}
\newtheorem{proposition}{猜测}
\newtheorem{lemma}{引理}
\newtheorem{corollary}{推论}
\renewcommand{\proofname}{证明}
\renewcommand{\contentsname}{\center\hei{\sanhao{目录}}}
% \renewcommand{\refname}{\textbf{\xiaosi{\song{参考文献}}}}      % 将References改为参考文献

\newenvironment{chabstract}{{\hei{\xiaosi{摘要:}}}}            %定义中文摘要

\newenvironment{enabstract}{{\bfseries{\xiaosi\tnroman{Abstract:}}}}         %定义英文摘要

\newenvironment{chkeyword}{{\hei{\xiaosi{关键词:}}}}           %定义中文关键词

\newenvironment{enkeyword}{{\bfseries{\xiaosi\tnroman{Key words:}}}}         %定义英文关键词

\newcommand{\ud}{\mathrm{d}}                                    %用\ud 作为微分算子“d”

%%%%%%%%%%%%%%%%%%%%%%%%%%%%%%以上是一些命令或环境的重定义或自定义%%%%%%%%%%%%%%%%%%%%%%%%



\setCJKmonofont{SimSun}   % 设置等宽字体
\setmainfont{Times New Roman} %设置默认英文字体。
%%%%%%%%%%%%%%%%%%%%%%%%%以上为中英文字体设置%%%%%%%%%%%%%%%%%%%%%%%%%%%%%%%%%%%%%%%%%%%




%%%%%%%%%%%%%%%%%%%%%%%以下是版面控制部分%%%%%%%%%%%%%%%%%%%%%%%%%%%%%%%%%%%%%%%%%%%%%%
\usepackage{geometry}\geometry{left=2.75cm,right=2.5cm,top=2.5cm,bottom=2.5cm}
\usepackage{indentfirst}             % 首行缩进
\usepackage[perpage,symbol]{footmisc}% 脚注控制
\usepackage[sf]{titlesec}            % 控制标题
\usepackage{titletoc}                % 控制目录

\usepackage{graphicx}
\title{\hei{\sanhao{地图模块}}}
\author{\li{\xiaosi{忻斌健}}}
\date{}             %本文中手动添加时间

\begin{document}
\begin{titlepage}
	\centering
	\includegraphics[width=0.15\textwidth]{patac.jpg}\par\vspace{1cm}
	{\scshape\LARGE 泛亚汽车技术中心 \par}
	\vspace{1cm}
	{\scshape\Large 阶段性报告\par}
	\vspace{1.5cm}
	{\huge\bfseries 地图模块\par}
	\vspace{2cm}
	{\Large\itshape 忻斌健\par}
	\vfill
	协助\par
	丁稼毅\ 鞠一鸣\ 钱士才\ 李亚光

	\vfill

% Bottom of the page
	{\large \today\par}


\end{titlepage}

\clearpage                          %双面打印(openright) 用\cleardoublepage,刷新页面信息,为了添加目录章节后页码不乱
\pagenumbering{arabic}              %自此处页码开始计数
\addcontentsline{toc}{section}{\textbf{\xiaosi{摘要}}} %创建虚拟章节,便于将摘要部分添加到目录
\maketitle
\begin{chabstract}
本文讨论了地图模块的接口设计和主要的功能模块以及车辆和目标在地理坐标系下的定位$\cdots$.\\
\end{chabstract}

\begin{chkeyword}
{\hei\xiaosi{地图;RTK;数据融合; 路径规划;微波雷达;SRR, ESR.}}\\
\end{chkeyword}
%%%%%%%%%%%%%%%%%%%%%%%%%%%%%以上是中文摘要、关键词%%%%%%%%%%%%%%%%%%%%%%%%%%%%%%%%%



% 中图分类号:O177\\



%%%%%%%%%%%%%%%%%%%%%%%%%%%%%以下是英文题目、姓名、摘要、关键词%%%%%%%%%%%%%%%%%%%%%%%%%%%%%%%%%%%%%
\begin{center}{\bfseries{\sanhao\tnroman{HD Map implementation by RTK deployment for Path Planning with Vehicle and Object localization}}} \\
\end{center}

\begin{center}{\sihao\tnroman{Xin Binjian}}\\
\end{center}

\begin{enabstract}\ {\xiaosi\tnroman{This document has discussed the implementation of HD Map and the vehicle and objects localization in geographic coordinate system with RTK.\\}}
\end{enabstract}

\begin{enkeyword}
\ {\xiaosi{HD Map;\ \ RTK;\ \ Sensor Fusion;\ \ isomorphism;\ \ Fourier transform.}}\\
\end{enkeyword}
%%%%%%%%%%%%%%%%%%%%%%%%%%%%%以上是英文题目、姓名、摘要、关键词%%%%%%%%%%%%%%%%%%%%%%%%%%%%%%%
\newpage

% \li{这是隶书}\\
% \consol{注释 test}

%%%%%%%%%%%%%%%%%%%%%%%%%%%%%以下为论文引言部分%%%%%%%%%%%%%%%%%%%%%%%%%%%%%%%%%%%%%%%%%
\section{\textbf{\xiaosi{引言}}}

{\xiaosi{\song{在一般的本科生泛函分析教材中, $\cdots$ 如~$L^p(\mathbf{E})$~和~$l^p$~}}}\\
%%%%%%%%%%%%%%%%%%%%%%%%%%%%%以上为论文引言部分%%%%%%%%%%%%%%%%%%%%%%%%%%%%%%%%%%%%%%%%%%

\color{} 
{\xiaosi{\song{

%%%%%%%%%%%%%%%%%%%%%%%%%%%%%以下为论文第二部分%%%%%%%%%%%%%%%%%%%%%%%%%%%%%%%%%%%%%%%%%%
\section{\textbf{\xiaosi 数列空间和函数空间的定义}}

以下给出六种典型的数列空间和函数空间的定义,文字叙述和符号表示依照文献\cite{1}.\\


\begin{definition}[空间~$l^p$   $(p\geq 1)$]               %%%%%%%%%%%%%%定义 1%%%%%%%%%%%
一切满足
~$(\sum\limits^{\infty}_{i=1}|\xi_i|^p)^{1/p}<+\infty$~    %注意命令 \limits
的数列~$x=(\xi_1,\xi_2,\cdots)$~的全体记为~$l^p$.\ 容易验证
$${\parallel x\parallel}_{p}=(\sum^{\infty}_{i=1}|\xi_i|^p)^{1/p}<+\infty$$ 是~$l^p$~上的范数.
\end{definition}
$\cdots$
%%%%%%%%%%%%%%%%%%%%%%%%以上为论文正文第二部分%%%%%%%%%%%%%%%%%%%%%%%%%%%%%%%%%%%%%%%%%%%%%%%



%%%%%%%%%%%%%%%%%%%%%%%%以下为论文正文第三部分%%%%%%%%%%%%%%%%%%%%%%%%%%%%%%%%%%%%%%%%%%%%%%
\section{\textbf{\song{\xiaosi 六类空间各自的性质}}}

{\subsection{\textbf{\song{\xiaosi{$l^p$和$L^p(\mathbf{E})$}}}}}
$l^p$~和~$L^p(\mathbf{E})$~都可分.
$\cdots$
%%%%%%%%%%%%%%%%%%%%%%%%%%%以上为论文正文第三部分%%%%%%%%%%%%%%%%%%%%%%%%%%%%%%%%%%%%%%%%%%




%%%%%%%%%%%%%%%%%%%%%%%%%%%以下为论文正文第四部分%%%%%%%%%%%%%%%%%%%%%%%%%%%%%%%%%%%%%%%%%%
{\section{\textbf{\song{\xiaosi 六种空间之间的一些联系}}}}

{\subsection{\textbf{\song\xiaosi{函数空间与函数空间、数列空间与数列空间之间的联系}}}}


$\cdots$
{\subsection{\textbf{\song\xiaosi{函数空间与数列空间的联系}}}}

\begin{lemma}[~Riesz-Fiesher~定理]                %%%%%%%%%%%%%%%%%%%%%% 引理 1 %%%%%%%%%%%%%%%%%%%%%%%%
设~$\{e_n\}$~是~Hilbert~空间~$\mathbf{H}$~中一就范正交系,$(c_1,c_2,\cdots)\in l^2$,则存在唯一的~$x\in H$~使~$(x,e_n)=e_n,\quad n=1,2,\cdots$~并且~$(x,x)=\sum\limits_{n=1}^\infty|c_n|^2$.\\
\end{lemma}


文献\cite{4}给出了~$1\leq p\leq 2$~时的~$L^p(\mathbf{E})$~上的~Fourier~变换的构造过程,并指出当
~$p>2$~时在广义函数的意义下~$L^p(\mathbf{E})$~仍可导入~Fourier~变换.\ 问题在于~$p\neq 2$~时~Fourier
~变换能否构成~$L^p(\mathbf{E})$~与~$l^p$~之间的保范同构.


\begin{property}\label{pro2}                  %%%%%%%%%%%%%%%%% 问题 2 %%%%%%%%%%%%%%%%%%%%%%%%
完备距离空间~$S(\mathbf{E})$~与~$s$,Banach~空间~$M(\mathbf{E})$~与~$m$~之间是否有同构关系?更进一步,~Fourier~变换及其反演公式
能否推广到完备距离空间~$S(\mathbf{E})$~与~$s$,Banach~空间~$M(\mathbf{E})$~ 与~$m$
\end{property}
%%%%%%%%%%%%%%%%%%%%%%%%%以上是论文正文的第四部分%%%%%%%%%%%%%%%%%%%%%%%%%%%%%%%%%%%%%%%%%%%%
}}}


%%%%%%%%%%%%%%%%%%%%%%%%%以下是参考文献%%%%%%%%%%%%%%%%%%%%%%%%%%%%%%%%%%%%%%%%%%%%%%%%

\clearpage %双面打印(openright) 用\cleardoublepage
\addcontentsline{toc}{section}{\textbf{\xiaosi{参考文献}}}
\begin{thebibliography}{99}
{\song{\wuhao
\bibitem{1}那汤松.\ 实变函数论(第5版).\ 徐瑞云 译.\ 北京:高等教育出版社,2010.
\bibitem{2}郭大钧等.\ 实变函数与泛函分析(第二版)$\cdot$ 下册.\ 山东:山东大学出版社,2005.
\bibitem{3}夏道行等.\ 实变函数论与泛函分析(下册$\cdot$ 第二版修订本).\ 北京:高等教育出版社,2010.
\bibitem{4}A.H.柯尔莫戈洛夫,C.B.佛明.\ 函数论与泛函分析初步(第7 版).\ 北京:高等教育出版社,2006.
}
}
\end{thebibliography}

\end{document}